\documentclass[11pt,a4paper,oneside]{article}

\usepackage{amsmath}
\title{Breaking a Predictable ECDSA Nonce}
\date{\today}
\author{Jonas Nick}

\begin{document}
\maketitle 
Let $n, G$ be the parameters of secp256k1, where $n$ is the curve order and $G$ is the base point.
Let $d_A$ be the private key $\in [1, n-1]$, $z$ be the hash of the message, $(r,s)$ the signature corresponding
to the private key $d_A$ where $r,s \in [0, n-1]$, and $k$ the corresponding nonce.

Then it holds that $s=k^{-1}(z+rd_A)\mod n$. The obscuren/secp256k1-go package chooses $k$ to be
$z\oplus d_A$ (xor). 
At first glance this seems to be ok, since $k$ is unique for each message and
it is unpredictable. An attacker can not directly influence $z$ because it is the outcome of a hash function.
However, if an attacker obtains multiple signatures, the reuse of $d_A$ becomes a problem because $k$ 
becomes predictable. Note that $a \oplus b = a + b - 2(a\wedge b)$.
\begin{align*}
s&=k^{-1}(z+rd_A) \\
    &= (d_A\oplus z)^{-1}(z+rd_A)\\
    &= (d_A + z - 2(d_A\wedge b))(z+rd_A)\\
    \iff d_As + zs - 2s(d_A\wedge z)&= z + rd_A\\
    \iff (s-r)d_A &= (1-s)z + 2s(d_A\wedge z)\\
    \iff d_A &= ((1-s)z + 2s(d_A\wedge z))(s-r)^{-1}
\end{align*}
Assume that the attacker obtains a second signature $(r', s')$ over $z'$.
Then it holds that
\begin{align*}
    d_A - d_A &= ((1-s)z + 2s(d_A\wedge z))(s-r)^{-1} - ((1-s')z' + 2s'(d_A\wedge z'))(s'-r')^{-1} \\
\iff  0  &= (1-s)z(s-r)^{-1} + 2s(d_A\wedge z)(s-r)^{-1} - (1-s')z'(s'-r')^{-1} - 2s'(d_A\wedge z')(s'-r')^{-1} 
\end{align*}
$$
    \iff(1-s')z'(s'-r')^{-1} - (1-s)z(s-r)^{-1}=  2s(s-r)^{-1}(z \wedge d_A) - 2s'(s'-r')^{-1}(z' \wedge d_A) ) \\
$$


Let 
\begin{align*}
    \alpha &= (1-s')z'(s'-r')^{-1} - (1-s)z(s-r)^{-1}, \\
    \beta &= 2s(s-r)^{-1}, \\
    \gamma &= 2s'(s'-r')^{-1} 
\end{align*}

such that

\begin{align*}
    \alpha &= \beta (d_A \wedge z) - \gamma(d_A \wedge z) \\
           &= \sum_{i} \beta_i 2^i\sum_{j} d_{A_j} z_j 2^{j} - \sum_{i} \gamma_i' 2^i\sum_{j} d_{A_j} z_j' 2^{j} \\
    &= \sum_{j} d_{A_j} 2^j \sum_i (\beta_i z_j - \gamma_i' z_j') 2^i \\
    &= \sum_{j} d_{A_j} 2^j (z_j \beta - z_j' \gamma')
\end{align*}
where $\delta_i$ is the $i$-th bit of some $\delta$.
Now the attacker collects 256 such signature pairs and solve the linear system for $d_A$ with $2^j (z_j k - z_j' k')$ as coefficient.

\end{document}

