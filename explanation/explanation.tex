\documentclass[11pt,a4paper,oneside]{article}

\usepackage{amsmath}
\title{Breaking a Predictable ECDSA Nonce}
\date{\today}
\author{Jonas Nick}

\begin{document}
\maketitle 
Let $n, G$ be the parameters of secp256k1, where $n$ is the curve order and $G$ is the base point.
Let $d$ be the private key $\in [1, n-1]$, $z$ be the hash of the message, $(r,s)$ the signature corresponding
to the private key $d$ where $r,s \in [0, n-1]$, and $k$ the corresponding nonce.

Then it holds that $s=k^{-1}(z+rd)\mod n$. The github.com/obscuren/secp256k1-go package chooses $k$ to be
$z\oplus d$ (xor). 
At first glance this seems to be ok, since $k$ is unique for each message and
it is unpredictable. An attacker can not directly influence $z$ because it is the outcome of a hash function.
However, if an attacker obtains multiple signatures, the reuse of $d$ becomes a problem because $k$ 
becomes in fact predictable. 

The problem can be reformulated to a linear system:
\begin{equation}
    \alpha = \sum_i d_i 2^i \beta_i
\end{equation}
where $\alpha = (s-1)z$ and $\beta_i = (r + (2z_i - 1)s)$ and $d_i$, $z_i$ are the $i$-th bit in the binary representation of $d$ and $z_i$.
Thus, the attacker collects 256 signatures and solves the linear system for $d$. 
In other words, each signature leaks one bit of the private key.

\section{Proof}

Note that $a \oplus b = a + b - 2(a\wedge b)$.
\begin{align*}
s&=k^{-1}(z+rd) \\
    &= (d\oplus z)^{-1}(z+rd)\\
    &= (d + z - 2(d\wedge b))^{-1}(z+rd)\\
    \iff ds + zs - 2s(d\wedge z)&= z + rd\\
    \iff (s-1)z &= 2s(d\wedge z) (s-r)d\\
                &= \sum_i 2^i d_i z_i 2s + \sum_i 2^i d_i (r-s) \\
                &= \sum_i d_i 2^i (r + (2z_i - 1)s) \\
\end{align*}
q.e.d
\end{document}

